\section{2. Sitzung}
\label{sec:2_sitzung}

\textbf{Datum}: 26.04.2023\\
\textbf{Schriftführer}: Martin Steiner\\
\textbf{Anwesenheit}: Max Jost, Martin Steiner, Aleksey Sokolov

\subsection*{Sitzungsprotokoll}

Die Ergebnisse der Arbeitsaufträge der vorherigen Woche wurden präsentiert und besprochen:
\begin{itemize}
    \item \textbf{Max Jost:} \textit{"Recherche der Funktionsweise und Ansteuerungsart der Steppermotoren, bzw. welche Motoren für den Antrieb der Haupt- und Nebenachse in Frage kommen."}\\
    Steppermotoren für beide Achsen scheinen die beste Lösung zu sein. Art und Preis abhängig vom benötigten Drehmoment. Für die Ansteuerung wird ein Driver benötigt; Driver mit Interpolation wäre besser, kostet minimal mehr.
    \item \textbf{Aleksey Sokolov:} \textit{"Recherche der Funktionsweise möglicher Drahtspannsysteme und deren Praktikabilität für das Projekt."}\\
    Vorstellung zweier möglicher Spannsysteme. Beide Systeme verwenden einen Abrollmotor und einen zusätzlichen Motor zur Feinspannung des Drahtes. Die Messung der Spannung wird beim zweiten System über die Winkelauslenkung eins federgestärkten Dancingarms realisiert.
    \item \textbf{Martin Steiner:} \textit{"Recherche über die unterschiedlichen Linearführungsarten und Antriebsvarianten, sowie Beschaffbarkeit und Preisvergleich."}\\
    Für die Linearführung kommen drei Arten in Frage, wobei aus Preis- und Verfügbarkeitsgründen Wellenführungen vorzuziehen sind. Für den Antrieb scheint, aufgrund der geforderten Genauigkeit, eine Kugelumlaufspindel die beste Wahl zu sein. Preise und Lieferanten wurden recherchiert. Ein Excel-Dokument zur Kostenkalkulation Erstellt und den Teampartnern zugänglich gemacht.
\end{itemize}

Die Konstruktionsweise für das Spannsystems wurde nach einer Diskussion festgelegt. Es soll eine konstante Gegenkraft an der Abrollseite, mittels eines Filzklemmsystems, realisiert werden. Mit Hilfe eines Dancingarms, mit befestigten Permanentmagneten, welche durch einen Elektromagneten gesteuert werden kann, sollen dynamische Spannungsänderungen des Drahtes nachkorrigiert werden können. Ansteuerung mittels eines Konstant-Strom-Regelkreises. Zusätzliche Drahtspannungsmessung eventuell durch Piezosensor, oder Winkelauslenkung des Dancingarms.\\

% #TODO:Martin - Skizze einfügen

Es wurden neue Arbeitsaufträge bis zur nächsten Sitzung vereinbart:

\begin{itemize}
    \item \textbf{Max Jost:} Berechnung der, für die Steppermotorenauswahl, relevanten mechanischen Parameterintervalle (Drehmoment, etc.). Gemeinsam mit Aleksey Sokolov.
    \item \textbf{Aleksey Sokolov:} Berechnung der, für die Steppermotorenauswahl, relevanten mechanischen Parameterintervalle (Drehmoment, etc.). Gemeinsam mit Max Jost.\\
    Freiwillige Meldung zur Überlegung der Wirkenden Kräfte im festgelegten Spannsystems und deren mathematische Modellierung (sofern zeitlich machbar).
    \item \textbf{Martin Steiner:} Skizzenhafte Planung und Zeichnung des Coilwinders. Abschätzung der Masse des Drahtführungstisches. Rücksprache und Kontrolle der Konstruktionsideen mit Maschinebau-Cousin.
\end{itemize}