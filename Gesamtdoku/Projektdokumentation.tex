%---------------DOCUMENT SETTINGS------------------------------------
%\documentclass[headheight=30pt]{scrartcl}
\documentclass{article}

\usepackage[utf8]{inputenc} % utf8x durch utf8 ersetzt wegen biblatex
\usepackage[T1]{fontenc}
\usepackage{amsmath,amssymb,amstext,amsfonts}
\usepackage[ngerman]{babel}
\usepackage{csquotes}
\usepackage{pdfpages} %zum importieren des Deckblattes
\usepackage{geometry}
\usepackage[headsepline]{scrlayer-scrpage}
\usepackage{lastpage}
\usepackage[style=numeric]{biblatex} %Literaturverwaltung
\usepackage{tabularx} %für die Legende im Gruundlagen Oszi Bild
\usepackage{float}
\usepackage{textcomp}
\usepackage{gensymb}
\usepackage{stmaryrd}
\usepackage{graphicx}
\usepackage{wrapfig}
\usepackage[export]{adjustbox}
\usepackage{a4wide}
\usepackage{hyperref}
\usepackage{multicol}
\usepackage{makecell}
\usepackage{enumitem}
\usepackage{subcaption}
\usepackage[official]{eurosym}
\usepackage{siunitx}    % SI-Units einbindung

\geometry{      %holt mehr aus einer A4 Seite raus
    a4paper,
    total={170mm,257mm},
    left=20mm,
    top=25mm,
   }

% \geometry{      %holt noch um einiges mehr aus einer A4 Seite raus
%     a4paper,
%     total={170mm,257mm},
%     left=10mm,
%     right=10mm,
%     top=10mm,
%     bottom=10mm,
%    }

\graphicspath{ {./graphics/} }  %pfad für bilder
\hypersetup{colorlinks=false}
\addbibresource{literatur.bib} %Literatur-Resourcen\s
%\setlength{\headheight}{0.0pt} % macht zwar headheight warnings aber dafür nutz latex die seitengröße besser aus.
\pagestyle{empty}


\newcommand{\subf}[2]{
    {
        \begin{tabular}[c]{@{}c@{}}
            {\setlength{\extrarowheight}{100pt} #1 }\\#2
        \end{tabular}
    }
}
\newcommand{\allc}{\multicolumn{1}{c|}{-}}
\newcommand{\monofig}[4]{
    
    \begin{figure}[H]
    \centering
    \includegraphics[#1]{#2}
    \caption{
        #3
    }
    \label{#4}
    \end{figure}
    
}
\newcommand{\polyfig}[6]{
    \begin{figure}[H]
        \centering
        \begin{minipage}[b]{0.45\textwidth}
            \centering
            \includegraphics[width=\textwidth]{#1}
            \caption{#2}
            \label{#3}
        \end{minipage}
        \begin{minipage}[b]{0.45\textwidth}
            \centering
            \includegraphics[width=\textwidth]{#4}
            \caption{#5}
            \label{#6}
        \end{minipage}
    \end{figure}
}
\newcommand{\polyfigacc}[8]{
    \begin{figure}[H]
        \centering
        \begin{minipage}[b]{#1}
            \centering
            \includegraphics[width=\textwidth]{#2}
            \caption{#3}
            \label{#4}
        \end{minipage}
        \begin{minipage}[b]{#5}
            \centering
            \includegraphics[width=\textwidth]{#6}
            \caption{#7}
            \label{#8}
        \end{minipage}
    \end{figure}
}
\newcommand{\trifig}[9]{
    \begin{figure}[H]
        \centering
        \begin{minipage}[b]{0.32\textwidth}
            \centering
            \includegraphics[width=\textwidth]{#1}
            \caption{#2}
            \label{#3}
        \end{minipage}
        \begin{minipage}[b]{0.32\textwidth}
            \centering
            \includegraphics[width=\textwidth]{#4}
            \caption{#5}
            \label{#6}
        \end{minipage}
        \begin{minipage}[b]{0.32\textwidth}
            \centering
            \includegraphics[width=\textwidth]{#7}
            \caption{#8}
            \label{#9}
        \end{minipage}
        \caption{#10}
        \label{#11}
    \end{figure}
}
\newcommand{\trifigcom}[8]{
    \begin{figure}[H]
        \centering
        \begin{subfigure}[b]{0.32\textwidth}
            \centering
            \includegraphics[width=\textwidth]{#1}
            \caption{}
            \label{#2}
        \end{subfigure}
        \begin{subfigure}[b]{0.32\textwidth}
            \centering
            \includegraphics[width=\textwidth]{#3}
            \caption{}
            \label{#4}
        \end{subfigure}
        \begin{subfigure}[b]{0.32\textwidth}
            \centering
            \includegraphics[width=\textwidth]{#5}
            \caption{}
            \label{#6}
        \end{subfigure}
        \caption{#7}
        \label{#8}
    \end{figure}
}

%\date{\vspace{-1cm}\today{}, Graz}
\date{}
\author{Martin Steiner, Max Jost}
\title{\vspace{-1.5cm}Automatic Coil Winder}

% ---------------HEADER TEXT------------------------------------
\clearpairofpagestyles
\ihead{\today{} \\ }
\chead{Max Jost / Martin Steiner\\ Projektname }
\ohead{CES \\ }
\cfoot{\pagemark \, / \, \pageref{LastPage}}

%---------------DOCUMENT TEXT------------------------------------
\begin{document}
    % \begin{multicols}{2}
    %     \maketitle
    %     \vspace{-1cm}
    %     \par\noindent\rule{\linewidth}{0.4pt}
    %     \\
    %     \\
    %     \section{Aufgabenstellung}
\label{sec:Aufgabenstellung}
    %     \section{Physikalisches Modell}
\label{sec:Physikalisches Modell}

\begin{itemize}
    \item Was für ein System wollen Sie betrachten, was für Gesetzmäßigkeiten liegen vor?
    \item Welche Formeln sollten gültig sein?
    \item Wie könnte das System modelliert werden?
    \item Welche möglichen Störgrößen würden Sie erwarten?
\end{itemize}


    %     \section{Hardware}
\label{sec:Hardware}





\subsection{Sensoren}

Zur Messung der Drahtspannungskraft wird eine Wägezelle verwendet, mit Messbereich 0-1kg verwendet. Der Sensor besteht aus einem Metallblock mit zwei Bohrungen. Auf der ober und unter Seite des Blocks sind jeweils zwei Dehnmessstreifen einer 'Wheatstone Bridge' befestigt. Wird der Sensor durch Krafteinwirkung gebogen, so kann eine Spannungsänderung an den zwei abgegriffen Punkten $A^+$ und $A^-$ (siehe \autoref{fig:schaltplan}) eine Änderung der Spannung gemessen werden. Diese verhält sich im Messbereich annähernd linear. Durch Austesten konnte festgestellt werden, dass die Auflösung bei ca. 0.5~g liegt. Die Kalibrierung des Sensors wird in \autoref{sec:Messablauf} beschrieben.\newline
Ausgelesen wird der Sensor mittels eines 24-Bit-ADC´s am HX711-Board, von welchem die Daten danach über eine Serielleschnittstelle auf den Microcontroller und danach per USB an den Computer übertragen werden. Eine genauere Beschreibung der technischen Messumsetzung ist in \autoref{sec:Messschleife} ersichtlich. 


Weiters wurden zwei Endstopps verwendet. Diese sind einfache Taster welche ein Signal

% #TODO: schreiben
\subsection{Aktoren}

Als Aktoren wurden zwei ACT 24HS5430D8L2 NEMA23 Steppermotoren verwendet, da diese das für uns notwendige Haltemoment von 150~$N \cdot cm$ bieten. Das Funktionsprinzip eines Steppermotors ist die Ausrichtung eines Stators durch Erzeugung geeigneter Magnetfelder, mithilfe zweier Spulen. Der Aufbau des Motors ist auf der rechten Seite von \autoref{fig:schaltplan} ersichtlich. \newline
Zur Ansteuerung der Motoren wurden zwei TMC2209 Stepperdriver verwendet, welche zwar laut Datenblatt 2,5 A liefern können, von uns aus Gründen der Überhitzung nur bei ca. 1,7 A Belastung betrieben werden. Die vom Motor nutzbaren 3~$A~Phase^{-1}$ können somit noch nicht voll ausgeschöpft werden. Grund dafür ist eine Lieferverzögerung von aktuell ca. 2 Monaten der eigentlichen Treiber.


Weitere Bauteile sind:
\begin{itemize}
    \item HX711
    \item Arduino UNO (ATMEGA328p)
    \item Netzteil
    \item Stepdownconverter
\end{itemize}

\begin{figure}[H]
    \centering
    \includegraphics[width=1\textwidth]{./schaltplan.png}
    \caption{Schaltplan}
    \label{fig:schaltplan}
\end{figure}

In \autoref{fig:schaltplan} ist die elektronische Verschaltung des gesamten Projekt schematisch dargestellt. Der eingezeichnete Microcontroller, beschriftet mit 'USB', ist der Oben bereits erwähnte Arduino Uno. Die zwei, neben dem Lineartisch, eingezeichneten, und mit dem Microcontroller verschaltenen, Bauteile, sind die zwei Endstopps, welche im 'Normaly Open'-Modus $NO$ betrieben werden, um im Falle eines Kabelbruches einem Maschinenschaden vorzubeugen.
    %     \section{Mechanik}
\label{sec:Mechanik}

Wie ist das System realisiert?

Das Projekt teilt sich in zwei mechanische Untersyteme auf, welchex
    %     \section{Sensoren und Akktoren}
\label{sec:Sensoren_Akktoren}

\begin{itemize}
    \item  Bitte keine Copy+Paste Texte vom Hersteller in Text oder Präsentation
    \item  Kurz und knackig, was kann der Sensor
    \item  Was ist das physikalische Messprinzip
    \item  Sensor/Aktor Kalibrierung
        \begin{itemize}
            \item Wie haben Sie gewährleistet, dass der Sensor richtig misst?
            \item Was ist ihr Normativ?
            \item Wie hoch ist die Standardabweichung des Sensors im Ruhezustand mit/ohne 
            System?
        \end{itemize}    
\end{itemize}
    %     \section{Messablauf}
\label{sec:Messablauf}

% Wie läuft die Messung ab? (Ereignis/Kontinuierlich ect.)
% Was für eine Messprozedur wird verwendet?


Bestimmung des realen Skalierungsfaktors beschreiben sie kapitel pyhsik

Für die Experimente wurde aus Kostengründen Nähgarn verwendet mit einem Durchmesser $D = 0,25~\si{\milli\metre}$ verwendet. Daraus ergibt sich nach \autoref{eq:achsenskalierungsfaktor} ein Skalierungsfaktor von $K_{xy} = 4$.Das zuwickelnde Medium (Faden, Draht, etc.) wird dann durch das Spannsystem und die Maschine eingeführt und am Spulenkörper befestigt. Die Maschine wird, mittels Commandlinebefehlen im Userinterface (UI), an ihre Startposition gefahren. Danach wird der Befehl zum Start des Kalibrationsscriptes gegeben, woraufhin das Programm einen zum tarieren der Wägezelle, sowie zum Messen eines Referenzgewichtes, auffordert. Danach können die Befehl zur Durchführung unterschiedlicher Wickeloperationen gesendet werden. Es wird eine halbe Testlage gewickelt um zu sehen ob der Skalierungsfaktor stimmt und wenn nötig nachgestellt und erneut getestet. Für unseren Draht ergab sich somit ein Skalierungsfaktor von $K_{xy} = 3$, mit $C = 0,75$. Danach wird die  Messdatenübertragung per UI gestartet, wodurch die Übertragung der Messwerte über die Serielleschnittstelle beginnt. Bis zur Beendigung der Messung durch einen Commandlinebefehlen werden kontinuierlich Messdaten an den Computer übertragen. Für den genauen technischen Ablauf der Messung, siehe \autoref{sec:Messschleife}. 

Anmerken wie genau das mit der Kalibration der Wägezelle war, bzw. da wir ja nur an relativ Werten interessiert sind, der schwankende Offset irrelevant ist (z.B. wegen Temperatur).
    %     \section{Messschleife}
\label{sec:Messschleife}

\begin{itemize}
    \item Wie oft wird gesampelt?
    \item Wie wird gemittelt?
    \item Wie lange dauert die ganze Messschleife?
    \item Wie lange dauern die Messzeiten der einzelnen SW-Blöcke?
\end{itemize}
    %     \section{Code}
\label{sec:Code}
    %     \section{Messdaten}
\label{sec:Messdaten}

Für die Durchführung der Experimente wurden zwei Zylinder mit gleicher Höhe $h=50~\si{\milli\metre}$, aber unterschiedlicher Grundfläche, als Spulenkörper verwendet. Die Grundfläche der einen Spule ($SP_K$) entspricht einem Kreis mit Radius $r=10~\si{\milli\metre}$, die der zweiten Spule einem Quadrat ($SP_Q$) mit Seitenlänge $s=20~\si{\milli\metre}$. Die Experimente werden in \autoref{sec:Auswertung} näher ausgeführt. Zur beispielhaften Darstellung der Rohdaten ist hier der, absichtlich herbeigeführte, Abriss des Drahtes während einer Wickeloperation dargestellt.

% \begin{figure}[H]
%     \centering
%     \includegraphics[width=0.25\textwidth]{./winder_render.png}
%     \caption{a nice plot}
%     \label{fig:winder_render}
% % \end{figure}
% \begin{wrapfigure}{l}{0.35\textwidth}
%     \centering
%     \includegraphics[width=0.35\textwidth]{./spannplatte_render.jpg}
% \end{wrapfigure}
    %     \section{Auswertung}
\label{sec:Auswertung}

Wurde im Folgenden ein Versuch ohne Dancerarm durchgeführt, so wurde der Draht direkt über das Kugellager der darunter liegenden Leiste geführt. Des weiteren gilt für alle folgenden Graphiken, dass die Unsicherheitsbalken stets die einfache Standardabweichung angebegen.\newline
Wickelt man die selbe Wicklung, auf der Spule $SP_K$, zweimal mit unterschiedlicher Geschwindigkeit, so wird der, bereits in \autoref{sec:Physikalisches Modell} erwähnte, Effekt einer geschwindigkeitsabhängigen Drahtspannung , bzw. Rückstellkraft $F_R$ sichtbar. Der Versuch wurde je einmal mit Dancerarm und einmal ohne Dancerarm durchgeführt.

\begin{figure}[H]
    \centering
    \includegraphics[width=0.9\textwidth]{./const_speed.pdf}
    \caption{Darstellung der an der Wägezelle gemessenen Kraft während einer Wickelphase mit vorgegebener, konstanter Geschwindigkeit, $F_{WZ}$ in Abhängigkeit von der Winkelgeschwindigkeit der Hauptachse $\omega_{HA}$. Die, mit Unsicherheitsbalken versehenen, vier Datenpunkte beschreiben jeweils das arithmetische Mittel des angegeben Größe.}
    \label{fig:plot_const_speed}
\end{figure}


Eine mögliche Erklärung für die, in siehe \autoref{fig:plot_const_speed} ersichtliche, Geschwindigkeitsabhängigkeit der Drahtspannungskraft $\tau(v)$, bzw. Rückstellkraft $F_R(v)$, wäre, dass irgendwo im System eine Geschwindigkeitsabhängige Reibung vorkommt.
Geschwindigkeitsabhängige Reibungskräfte sind zum beispiel in der Aerodynamik als Luftwiderstand bekannt.
Unsere Annahme ist daher, dass entweder die Reibung, die durch das Öl in den V-Nut Kugellagern entsteht relativ groß ist, oder was wahrscheinlicher ist, dass die Reibungskraft des Filzes geschwindigkeitsabhängig ist.
Dies wiederum erklärten wir uns durch die Rückstellkräfte einzelner Filzfaser, die ähnlich auf den Faden drücken wie Fahrtwinddruck auf ein Auto.



Für den nächsten Versuch wurden die selbe Wicklung, bei gleichbleibenden Wickelparametern, einmal mit Dancerarm und einmal ohne, je für beide Spulenkörper, durchgeführt. Da die Zeit zwischen zwei Messwerten nicht konstant ist, sondern leicht schwankt (für Erklärung siehe \autoref{sec:Messschleife}), wurden die Daten Interpoliert, um äquidistante Messwertschritte zu erhalten. Danach wurde jeweils eine Fast-Fourier-Transformation $FFT$ durchgeführt und das Ergebnis in \autoref{fig:ffts} dargestellt.

\begin{figure}[H]
    \centering
    \includegraphics[width=0.9\textwidth]{./ffts.pdf}
    \caption{Darstellung der $FFTs$ für beide Spulenkörper, je einmal ohne Dancerarm und einmal mit Arm}
    \label{fig:ffts}
\end{figure}

Betrachtet man die zwei, $SP_K$, zugehörigen Graphen, in \autoref{fig:ffts}, so sieht man, dass es keinen signifikanten Unterschied gibt wenn der Dancerarm nicht verwendet wird.
Der Einfluss der, in \autoref{sec:Physikalisches Modell} vermuteten, abklingenden Schwingung, angeregt durch den Wechsel von Haft- zu Gleitreibung, scheint vergleichsweise wenig Einfluss auf die Schwingung der Drahtspannung zu haben.
Sieht man sich hingegen die zwei Graphen von $SP_Q$ an, so fallen sofort die deutlichen Peaks in der $FFT$ auf  welche auf das periodische Schwingen der Drahtspannung hindeuten. Der Peaks bei ca. 1 Hz ist auf die Unwucht der Hauptachse, der bei ca. 4 Hz Auf die 4 Ecken des Spulenkörpers zurückzuführen.Bei den restlichen Peaks dürfte es sich um Obertöne handeln, welche ganzzahlige Vielfache der Grundfrequenz 1 Hz sind.\newline

Zur Untersuchung der Start-, bzw. Abbremsphasen eines Wickeldurchganges sind in \autoref{fig:plot_beschleunigung} die zwei Phasen, je für beide Spulentypen dargestellt. Die Messungen wurde jeweils mit der selben Beschleunigung und Endgeschwindigkeit durchgeführt.


\begin{figure}[H]
    \centering
    \includegraphics[width=0.9\textwidth]{./besch.pdf}
    \caption{Darstellung der an der Wägezelle gemessenen Kraft $F_{WZ}$ gegen die Zeit $t$. Die Farbcodierung stellt die Geschwindigkeit der HA $\omega_{HA}$, von Gelb (große Geschwindigkeit) bis Lila (kleine Geschwindigkeit), dar. Für alle vier Graphen wurden die selben Wickelparameter verwendet. Die Messungen fanden ohne Dancerarm statt.}
    \label{fig:plot_beschleunigung}
\end{figure}

Sieht man sich die Graphiken der Bremsphasen an, so fällt auf, dass das System stark gedämpft ist. Betrachtet man die lilafärbigen Teile des Datensatzes, so sieht man nur einen relativen kleinen trägheitsbedingten Ausschwung der Kraft unter die anschließende Ruhelage. Die Filzklemme scheint also ihre im Modell (siehe \autoref{sec:Physikalisches Modell}) angedachte Funktion gut zu erfüllen. Im Bezug auf die Beschleunigungsphasen ist anzumerken, dass das die Kraft sehr schnell anfängt um einen Wert herum zu schwanken, obwohl die Beschleunigungsphase immer noch anhält. Dies wurde von uns ebenfalls nicht erwartet, da hier mit einer konstanten Beschleunigung gearbeitet wurde.  




    %     \section{Conclusio}
\label{sec:Conclusio}

    % \end{multicols}

\maketitle
\tableofcontents
\newpage

\section{Aufgabenstellung}
\label{sec:Aufgabenstellung}
\section{Mechanik}
\label{sec:Mechanik}

Wie ist das System realisiert?

Das Projekt teilt sich in zwei mechanische Untersyteme auf, welchex
\section{Hardware}
\label{sec:Hardware}





\subsection{Sensoren}

Zur Messung der Drahtspannungskraft wird eine Wägezelle verwendet, mit Messbereich 0-1kg verwendet. Der Sensor besteht aus einem Metallblock mit zwei Bohrungen. Auf der ober und unter Seite des Blocks sind jeweils zwei Dehnmessstreifen einer 'Wheatstone Bridge' befestigt. Wird der Sensor durch Krafteinwirkung gebogen, so kann eine Spannungsänderung an den zwei abgegriffen Punkten $A^+$ und $A^-$ (siehe \autoref{fig:schaltplan}) eine Änderung der Spannung gemessen werden. Diese verhält sich im Messbereich annähernd linear. Durch Austesten konnte festgestellt werden, dass die Auflösung bei ca. 0.5~g liegt. Die Kalibrierung des Sensors wird in \autoref{sec:Messablauf} beschrieben.\newline
Ausgelesen wird der Sensor mittels eines 24-Bit-ADC´s am HX711-Board, von welchem die Daten danach über eine Serielleschnittstelle auf den Microcontroller und danach per USB an den Computer übertragen werden. Eine genauere Beschreibung der technischen Messumsetzung ist in \autoref{sec:Messschleife} ersichtlich. 


Weiters wurden zwei Endstopps verwendet. Diese sind einfache Taster welche ein Signal

% #TODO: schreiben
\subsection{Aktoren}

Als Aktoren wurden zwei ACT 24HS5430D8L2 NEMA23 Steppermotoren verwendet, da diese das für uns notwendige Haltemoment von 150~$N \cdot cm$ bieten. Das Funktionsprinzip eines Steppermotors ist die Ausrichtung eines Stators durch Erzeugung geeigneter Magnetfelder, mithilfe zweier Spulen. Der Aufbau des Motors ist auf der rechten Seite von \autoref{fig:schaltplan} ersichtlich. \newline
Zur Ansteuerung der Motoren wurden zwei TMC2209 Stepperdriver verwendet, welche zwar laut Datenblatt 2,5 A liefern können, von uns aus Gründen der Überhitzung nur bei ca. 1,7 A Belastung betrieben werden. Die vom Motor nutzbaren 3~$A~Phase^{-1}$ können somit noch nicht voll ausgeschöpft werden. Grund dafür ist eine Lieferverzögerung von aktuell ca. 2 Monaten der eigentlichen Treiber.


Weitere Bauteile sind:
\begin{itemize}
    \item HX711
    \item Arduino UNO (ATMEGA328p)
    \item Netzteil
    \item Stepdownconverter
\end{itemize}

\begin{figure}[H]
    \centering
    \includegraphics[width=1\textwidth]{./schaltplan.png}
    \caption{Schaltplan}
    \label{fig:schaltplan}
\end{figure}

In \autoref{fig:schaltplan} ist die elektronische Verschaltung des gesamten Projekt schematisch dargestellt. Der eingezeichnete Microcontroller, beschriftet mit 'USB', ist der Oben bereits erwähnte Arduino Uno. Die zwei, neben dem Lineartisch, eingezeichneten, und mit dem Microcontroller verschaltenen, Bauteile, sind die zwei Endstopps, welche im 'Normaly Open'-Modus $NO$ betrieben werden, um im Falle eines Kabelbruches einem Maschinenschaden vorzubeugen.
\section{Physikalisches Modell}
\label{sec:Physikalisches Modell}

\begin{itemize}
    \item Was für ein System wollen Sie betrachten, was für Gesetzmäßigkeiten liegen vor?
    \item Welche Formeln sollten gültig sein?
    \item Wie könnte das System modelliert werden?
    \item Welche möglichen Störgrößen würden Sie erwarten?
\end{itemize}


\section{Messablauf}
\label{sec:Messablauf}

% Wie läuft die Messung ab? (Ereignis/Kontinuierlich ect.)
% Was für eine Messprozedur wird verwendet?


Bestimmung des realen Skalierungsfaktors beschreiben sie kapitel pyhsik

Für die Experimente wurde aus Kostengründen Nähgarn verwendet mit einem Durchmesser $D = 0,25~\si{\milli\metre}$ verwendet. Daraus ergibt sich nach \autoref{eq:achsenskalierungsfaktor} ein Skalierungsfaktor von $K_{xy} = 4$.Das zuwickelnde Medium (Faden, Draht, etc.) wird dann durch das Spannsystem und die Maschine eingeführt und am Spulenkörper befestigt. Die Maschine wird, mittels Commandlinebefehlen im Userinterface (UI), an ihre Startposition gefahren. Danach wird der Befehl zum Start des Kalibrationsscriptes gegeben, woraufhin das Programm einen zum tarieren der Wägezelle, sowie zum Messen eines Referenzgewichtes, auffordert. Danach können die Befehl zur Durchführung unterschiedlicher Wickeloperationen gesendet werden. Es wird eine halbe Testlage gewickelt um zu sehen ob der Skalierungsfaktor stimmt und wenn nötig nachgestellt und erneut getestet. Für unseren Draht ergab sich somit ein Skalierungsfaktor von $K_{xy} = 3$, mit $C = 0,75$. Danach wird die  Messdatenübertragung per UI gestartet, wodurch die Übertragung der Messwerte über die Serielleschnittstelle beginnt. Bis zur Beendigung der Messung durch einen Commandlinebefehlen werden kontinuierlich Messdaten an den Computer übertragen. Für den genauen technischen Ablauf der Messung, siehe \autoref{sec:Messschleife}. 

Anmerken wie genau das mit der Kalibration der Wägezelle war, bzw. da wir ja nur an relativ Werten interessiert sind, der schwankende Offset irrelevant ist (z.B. wegen Temperatur).
\section{Messschleife}
\label{sec:Messschleife}

\begin{itemize}
    \item Wie oft wird gesampelt?
    \item Wie wird gemittelt?
    \item Wie lange dauert die ganze Messschleife?
    \item Wie lange dauern die Messzeiten der einzelnen SW-Blöcke?
\end{itemize}
\section{Code}
\label{sec:Code}
\section{Messdaten}
\label{sec:Messdaten}

Für die Durchführung der Experimente wurden zwei Zylinder mit gleicher Höhe $h=50~\si{\milli\metre}$, aber unterschiedlicher Grundfläche, als Spulenkörper verwendet. Die Grundfläche der einen Spule ($SP_K$) entspricht einem Kreis mit Radius $r=10~\si{\milli\metre}$, die der zweiten Spule einem Quadrat ($SP_Q$) mit Seitenlänge $s=20~\si{\milli\metre}$. Die Experimente werden in \autoref{sec:Auswertung} näher ausgeführt. Zur beispielhaften Darstellung der Rohdaten ist hier der, absichtlich herbeigeführte, Abriss des Drahtes während einer Wickeloperation dargestellt.

% \begin{figure}[H]
%     \centering
%     \includegraphics[width=0.25\textwidth]{./winder_render.png}
%     \caption{a nice plot}
%     \label{fig:winder_render}
% % \end{figure}
% \begin{wrapfigure}{l}{0.35\textwidth}
%     \centering
%     \includegraphics[width=0.35\textwidth]{./spannplatte_render.jpg}
% \end{wrapfigure}
\section{Auswertung}
\label{sec:Auswertung}

Wurde im Folgenden ein Versuch ohne Dancerarm durchgeführt, so wurde der Draht direkt über das Kugellager der darunter liegenden Leiste geführt. Des weiteren gilt für alle folgenden Graphiken, dass die Unsicherheitsbalken stets die einfache Standardabweichung angebegen.\newline
Wickelt man die selbe Wicklung, auf der Spule $SP_K$, zweimal mit unterschiedlicher Geschwindigkeit, so wird der, bereits in \autoref{sec:Physikalisches Modell} erwähnte, Effekt einer geschwindigkeitsabhängigen Drahtspannung , bzw. Rückstellkraft $F_R$ sichtbar. Der Versuch wurde je einmal mit Dancerarm und einmal ohne Dancerarm durchgeführt.

\begin{figure}[H]
    \centering
    \includegraphics[width=0.9\textwidth]{./const_speed.pdf}
    \caption{Darstellung der an der Wägezelle gemessenen Kraft während einer Wickelphase mit vorgegebener, konstanter Geschwindigkeit, $F_{WZ}$ in Abhängigkeit von der Winkelgeschwindigkeit der Hauptachse $\omega_{HA}$. Die, mit Unsicherheitsbalken versehenen, vier Datenpunkte beschreiben jeweils das arithmetische Mittel des angegeben Größe.}
    \label{fig:plot_const_speed}
\end{figure}


Eine mögliche Erklärung für die, in siehe \autoref{fig:plot_const_speed} ersichtliche, Geschwindigkeitsabhängigkeit der Drahtspannungskraft $\tau(v)$, bzw. Rückstellkraft $F_R(v)$, wäre, dass irgendwo im System eine Geschwindigkeitsabhängige Reibung vorkommt.
Geschwindigkeitsabhängige Reibungskräfte sind zum beispiel in der Aerodynamik als Luftwiderstand bekannt.
Unsere Annahme ist daher, dass entweder die Reibung, die durch das Öl in den V-Nut Kugellagern entsteht relativ groß ist, oder was wahrscheinlicher ist, dass die Reibungskraft des Filzes geschwindigkeitsabhängig ist.
Dies wiederum erklärten wir uns durch die Rückstellkräfte einzelner Filzfaser, die ähnlich auf den Faden drücken wie Fahrtwinddruck auf ein Auto.



Für den nächsten Versuch wurden die selbe Wicklung, bei gleichbleibenden Wickelparametern, einmal mit Dancerarm und einmal ohne, je für beide Spulenkörper, durchgeführt. Da die Zeit zwischen zwei Messwerten nicht konstant ist, sondern leicht schwankt (für Erklärung siehe \autoref{sec:Messschleife}), wurden die Daten Interpoliert, um äquidistante Messwertschritte zu erhalten. Danach wurde jeweils eine Fast-Fourier-Transformation $FFT$ durchgeführt und das Ergebnis in \autoref{fig:ffts} dargestellt.

\begin{figure}[H]
    \centering
    \includegraphics[width=0.9\textwidth]{./ffts.pdf}
    \caption{Darstellung der $FFTs$ für beide Spulenkörper, je einmal ohne Dancerarm und einmal mit Arm}
    \label{fig:ffts}
\end{figure}

Betrachtet man die zwei, $SP_K$, zugehörigen Graphen, in \autoref{fig:ffts}, so sieht man, dass es keinen signifikanten Unterschied gibt wenn der Dancerarm nicht verwendet wird.
Der Einfluss der, in \autoref{sec:Physikalisches Modell} vermuteten, abklingenden Schwingung, angeregt durch den Wechsel von Haft- zu Gleitreibung, scheint vergleichsweise wenig Einfluss auf die Schwingung der Drahtspannung zu haben.
Sieht man sich hingegen die zwei Graphen von $SP_Q$ an, so fallen sofort die deutlichen Peaks in der $FFT$ auf  welche auf das periodische Schwingen der Drahtspannung hindeuten. Der Peaks bei ca. 1 Hz ist auf die Unwucht der Hauptachse, der bei ca. 4 Hz Auf die 4 Ecken des Spulenkörpers zurückzuführen.Bei den restlichen Peaks dürfte es sich um Obertöne handeln, welche ganzzahlige Vielfache der Grundfrequenz 1 Hz sind.\newline

Zur Untersuchung der Start-, bzw. Abbremsphasen eines Wickeldurchganges sind in \autoref{fig:plot_beschleunigung} die zwei Phasen, je für beide Spulentypen dargestellt. Die Messungen wurde jeweils mit der selben Beschleunigung und Endgeschwindigkeit durchgeführt.


\begin{figure}[H]
    \centering
    \includegraphics[width=0.9\textwidth]{./besch.pdf}
    \caption{Darstellung der an der Wägezelle gemessenen Kraft $F_{WZ}$ gegen die Zeit $t$. Die Farbcodierung stellt die Geschwindigkeit der HA $\omega_{HA}$, von Gelb (große Geschwindigkeit) bis Lila (kleine Geschwindigkeit), dar. Für alle vier Graphen wurden die selben Wickelparameter verwendet. Die Messungen fanden ohne Dancerarm statt.}
    \label{fig:plot_beschleunigung}
\end{figure}

Sieht man sich die Graphiken der Bremsphasen an, so fällt auf, dass das System stark gedämpft ist. Betrachtet man die lilafärbigen Teile des Datensatzes, so sieht man nur einen relativen kleinen trägheitsbedingten Ausschwung der Kraft unter die anschließende Ruhelage. Die Filzklemme scheint also ihre im Modell (siehe \autoref{sec:Physikalisches Modell}) angedachte Funktion gut zu erfüllen. Im Bezug auf die Beschleunigungsphasen ist anzumerken, dass das die Kraft sehr schnell anfängt um einen Wert herum zu schwanken, obwohl die Beschleunigungsphase immer noch anhält. Dies wurde von uns ebenfalls nicht erwartet, da hier mit einer konstanten Beschleunigung gearbeitet wurde.  




\section{Conclusio}
\label{sec:Conclusio}

\appendix

\section{This is an Apendix}
\label{appx:test}

\begin{figure}[H]
    \centering
    \includegraphics[width=0.9\textwidth]{./Drahtführungsdorn_schnitt.png}
    \caption{a nice plot}
    \label{fig:drahtfuehrungsdorn_schnitt}
\end{figure}

\begin{figure}[H]
    \centering
    \includegraphics[width=0.75\textwidth]{./Lineartisch.png}
    \caption{a nice plot}
    \label{fig:linearstisch}
\end{figure}

\begin{figure}[H]
    \centering
    \includegraphics[width=0.9\textwidth]{./Aufnahmedorn_schnitt.png}
    \caption{a nice plot}
    \label{fig:aufnahmedorn_schitt}
\end{figure}

\end{document}