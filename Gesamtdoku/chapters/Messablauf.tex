\section{Messablauf}
\label{sec:Messablauf}

% Wie läuft die Messung ab? (Ereignis/Kontinuierlich ect.)
% Was für eine Messprozedur wird verwendet?


Das zuwickelnde Medium (Faden, Draht, etc.) wird durch das Spannsystem und die Maschine eingeführt und am Spulenkörperbefestigt. Die Maschine wird, mittels Commandlinebefehlen im Userinterface (UI), an ihre Startposition gefahren. Danach wird der Befehl zum Start des Kalibrationsscriptes gegeben, woraufhin das Programm einen zum tarieren der Wägezelle, sowie zum Messen eines Referenzgewichtes, auffordert. Danach kann der Befehl zum Start des Windingscriptes gesendet werden. Dieses Starten die Datenübertragung über die Serielleschnittstelle an den Computer, danach beginnt der Wickelprozess. Bis zur Beendigung der Wicklung werden kontinuierlich Messdaten an den Computerübertragen. Für den genauen technischen Ablauf der Messung, siehe \autoref{sec:Messschleife}. 

Anmerken wie genau das mit der Kalibration der Wägezelle war, bzw. da wir ja nur an relativ Werten interessiert sind, der schwankende Offset irrelevant ist (z.B. wegen Temperatur).