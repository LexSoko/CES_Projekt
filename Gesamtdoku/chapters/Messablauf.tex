\section{Messablauf}
\label{sec:Messablauf}

% Wie läuft die Messung ab? (Ereignis/Kontinuierlich ect.)
% Was für eine Messprozedur wird verwendet?


Bestimmung des realen Skalierungsfaktors beschreiben sie kapitel pyhsik

Für die Experimente wurde aus Kostengründen Nähgarn verwendet mit einem Durchmesser $D = 0,25~\si{\milli\metre}$ verwendet. Daraus ergibt sich nach \autoref{eq:achsenskalierungsfaktor} ein Skalierungsfaktor von $K_{xy} = 4$.Das zuwickelnde Medium (Faden, Draht, etc.) wird dann durch das Spannsystem und die Maschine eingeführt und am Spulenkörper befestigt. Die Maschine wird, mittels Commandlinebefehlen im Userinterface (UI), an ihre Startposition gefahren. Danach wird der Befehl zum Start des Kalibrationsscriptes gegeben, woraufhin das Programm einen zum tarieren der Wägezelle, sowie zum Messen eines Referenzgewichtes, auffordert. Danach können die Befehl zur Durchführung unterschiedlicher Wickeloperationen gesendet werden. Es wird eine halbe Testlage gewickelt um zu sehen ob der Skalierungsfaktor stimmt und wenn nötig nachgestellt und erneut getestet. Für unseren Draht ergab sich somit ein Skalierungsfaktor von $K_{xy} = 3$, mit $C = 0,75$. Danach wird die  Messdatenübertragung per UI gestartet, wodurch die Übertragung der Messwerte über die Serielleschnittstelle beginnt. Bis zur Beendigung der Messung durch einen Commandlinebefehlen werden kontinuierlich Messdaten an den Computer übertragen. Für den genauen technischen Ablauf der Messung, siehe \autoref{sec:Messschleife}. 

Anmerken wie genau das mit der Kalibration der Wägezelle war, bzw. da wir ja nur an relativ Werten interessiert sind, der schwankende Offset irrelevant ist (z.B. wegen Temperatur).