\section{Messschleife}
\label{sec:Messschleife}

Kern der Messschleife ist die Step-Impuls gebende ISR (Interrupt Service Routine) der Enceladus Software.
Diese ISR wird vom 16 bit Hardware Timer 1 je nach Motorgeschwindigkeit in fixen Zeitabständen ausgelöst.
Die Zeitabstände $\Delta t_{ISR}$ sind dabei die Inverse Step-Impuls Frequenz.
Die zweite zeitliche Limitierung der Messschleife ist die Messdauer des HX711 Chips.
Dieser benötigt $\Delta t_{HX711}$ = 12.5 \si{\milli\second} für eine Messung der Wägezelle.
Da $\Delta t_{HX711}$ im Vergleich zu $\Delta t_{ISR}$ konstant ist, wurde dies als Messintervall gewählt.
Um dies zu erreichen wird in der ISR die Funktion \verb|loadcell.is_ready()| abgefragt, welche nur abfragt ob der HX711 DT pin auf HIGH steht.
Ist dies der Fall, wird der aktuelle Wägezellen Messwert, die Step-Positionen von Haupt und Nebenachse und die aktuelle Zeit mit \verb|millis()| über die Serielle Schnittstelle ohne jegliche Verarbeitung an den PC gesendet.
Dies führte aber zu einigen Limitierungen die durch Zeitdruck nichtmehr umgehbar waren.
Erstens führte das, zu einer maximal möglichen Motordrehzahl von 0.8 U/s, da bei schnelleren Drehzahlen $\Delta t_{HX711} > \Delta t_{ISR}$ war und dadurch Step-Impulse dadurch nicht mehr mit konstanter Frequenz an den Motor gesendet wurden.
Zweitens wird der Wert der \verb|millis()| Funktion während der ISR nicht Aktualisiert, somit entsteht eine Varianz im Messdaten Intervall $\Delta t_{mess}$ = 11 - 13 \si{\milli\second}.

