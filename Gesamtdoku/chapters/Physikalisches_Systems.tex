\section{Physikalisches Modell}
\label{sec:Physikalisches Modell}

% Welche Formeln sollten gültig sein?
%     \item Wie könnte das System modelliert werden?
%     \item Welche möglichen Störgrößen würden Sie erwarten?
% Was für ein System wollen Sie betrachten, was für Gesetzmäßigkeiten liegen vor?

Zylinderspule die parallel gewickelt wird; Draht kann als Raumkurve aufgefasst werden; daraus abgeleitet das Wineklgeschwindigkeit von HA und NA direkt prop. ist;

=> linearer Umrechnungsfaktor zwischen HA und NA
    Stellkonzept: schnellere Motor wird gesteuert, langsamere folgt mit faktor nachlesen

\begin{equation}
    \omega_{HA} = \omega_{NA} \cdot \alpha
\end{equation}

Spannsystem durch angenommen konstante Rückstellkraft der Filzpatschen; Federgetriebener Arm; theoretisch keine Auslenkung bei konst. geschwindigkeit, real schwingt er immer leicht; Feder als immer perfekt vertikal ausgelenkt aproximiert winkel vernachlässigt geschwindigkeitsabhängige Kraft HA-seitig; Draht durch WZ-Kugellager genau 180° Zug nach oben;

=> Kraft am WZ-Kugellager = 2 * Drahtspannung; Winkelungenauigkeiten wurden hier wieder vernachlässigt

\begin{equation}
    \label{eq:wz_kraft}
    F_{WZ} = 2 \cdot \tau
\end{equation}

=> KEINE ERKLÄRUNG BIS JETZT: bei höher kraft von motor biegt arm weiter durch.
UNBEDINGT ABCHECKEN OB DRAHTSPANNUNGSÄNDERUNG IN DATEN ERSICHTLICH ODER NUR ARMEFFEKT?????


Lagenabhängigkeit von Kraft


Annahme: Reibungen geschwindigkeitsunabhängige nur Gleit- und Rollreibung, Kräftegleichgewicht nicht geschwindigkeitsabhängige (auser nicht rotationssymetriescher Spulenkörper)

