\section{Sensoren und Akktoren}
\label{sec:Sensoren_Akktoren}

% \item  Bitte keine Copy+Paste Texte vom Hersteller in Text oder Präsentation
% Kurz und knackig, was kann der Sensor
% Was ist das physikalische Messprinzip
% Sensor/Aktor Kalibrierung
%     Wie haben Sie gewährleistet, dass der Sensor richtig misst?
%     Was ist ihr Normativ?
%     Wie hoch ist die Standardabweichung des Sensors im Ruhezustand mit/ohne System?
Wägezelle, Rotary Encoder, Stepperdriver, Steppermotor, Arduino???

\subsection{Wägezelle}
\label{subsec:sensor_waegezelle}

Basierend auf einer Wheatstone-Bridge mit Dehnmessstreifen.
\subsection{Rotary Encoder}
\label{subsec:sensor_rotary-encoder}

Optischer Rotary-Encoder
\subsection{Stepperdriver TMC2209}
\label{subsec:sensor_driver}


\subsection{Steppermotor}
\label{subsec:sensor_motor}

ACT 24HS5430D8L2

full step $1,8 \pm 5$ per step;
3 A/phase;
2,4 V;
150 N.cm Haltemoment;
350 $g \cdot cm^2$ Drehmoment;

\subsection{HX711????}
\label{subsec:sensor_hx711}