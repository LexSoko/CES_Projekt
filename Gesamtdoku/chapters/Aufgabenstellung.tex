\section{Aufgabenstellung}
\label{sec:Aufgabenstellung}

Bau eines automatischen Coil-Winders für Drahtdurchmesser bis zu AWG 42 klein. Spulenkörper mit kreisförmigen Querschnitt, als auch mit nicht rotationssymetrieschem Querschnitt, sollen sowohl mit paralleler Drahtführung mehrlagig gewickelt werden können. Für nicht rotationssymetriesche Spulenkörper muss eine automatische Drahtspannungsvorrichtung konstruiert werden, welche eine zuvor einstellbare Spannung aufrechterhält. Die Drahtspannung soll mittels einer Wägezelle indirekt gemessen werden und die Daten zur Analyse des Wicklungsablaufes verwendet werde.

Das Wavewinding und Scatterwinding wurde nicht umgesetzt, da die Zeit für die Testung und Verbesserung des Notwendigen Arms nicht gereicht hat und die Scattersoftware steuerungstechnisch, sowohl software-, wie auch hardwareseitig, zu kompliziert (in der gegebenen Zeit) war. Als Grundlage für die Motorsteuerung wurde das 'Enceladus'-Projekt von Jan Enenkel verwendet, welcher uns dies dankenswerterweise zur Verfügung gestellt hat. Im Gegensatz zum Projektvorschlag wurde der Dancerarm, des Spannsystems, nicht mithilfe eines Magneten, sondern mit einer Rückstellfeder, realisiert.