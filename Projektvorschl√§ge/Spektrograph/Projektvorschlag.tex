\section*{Fragestellung}
Wie gut können Ionen und Moleküle in Lösungen detektiert werden und wie genau kann deren Konzentration bestimmt werden mit einem Selbstgebauten automatisiertem Prismenspektrograph?

\section*{Projektaufbau}
Skizze und kurze Beschreibung

\section*{Sensoren, Aktoren, Elektronik und $\mu$C's}
\begin{enumerate}
    \item Steppermotoren für Optikeinstellungen
    \item Stepperdriver
    \item hochwertige Photodiode oder CCD Chip für Spektrale Intensitätsmessung
    \item Verstärkerschaltung
    \item Arduino <Model>
    \item RP4
\end{enumerate}

\section*{Physikalische Anforderungen}
\begin{enumerate}
    \item Welche Ionen sollen detektiert werden mit deren Spektrallinien
    \item notwendiger Wellenlängenbereich und Genauigkeitsanforderungen an Aktöre und Photodiode
\end{enumerate}

\section*{Software}
\begin{enumerate}
    \item steuerung
    \item Kalibration
    \item Telemetrie
    \item Datenanalyse
\end{enumerate}

\section*{Kostenabschätzung}

\section*{Aufwandsabschätzung}

\section*{Wermachtwas}